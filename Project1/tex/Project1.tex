\documentclass[reprint, english,notitlepage]{revtex4-1}  % defines the basic parameters of the document
% if you want a single-column, remove reprint

% allows special characters (including æøå)
\usepackage[utf8]{inputenc}
\usepackage [norsk]{babel} %if you write norwegian
%\usepackage[english]{babel}  %if you write english


%% note that you may need to download some of these packages manually, it depends on your setup.
%% I recommend downloading TeXMaker, because it includes a large library of the most common packages.

\usepackage{physics,amssymb}  % mathematical symbols (physics imports amsmath)
\usepackage{graphicx}         % include graphics such as plots
\usepackage{xcolor}           % set colors
\usepackage{hyperref}         % automagic cross-referencing (this is GODLIKE)
\usepackage{tikz}             % draw figures manually
\usepackage{listings}         % display code
\usepackage{subfigure}        % imports a lot of cool and useful figure commands
\usepackage{verbatim}
\usepackage{adjustbox}


% defines the color of hyperref objects
% Blending two colors:  blue!80!black  =  80% blue and 20% black
\hypersetup{ % this is just my personal choice, feel free to change things
    colorlinks,
    linkcolor={red!50!black},
    citecolor={blue!50!black},
    urlcolor={blue!80!black}}

%% Defines the style of the programming listing
%% This is actually my personal template, go ahead and change stuff if you want
\lstset{ %
	inputpath=,
	backgroundcolor=\color{white!88!black},
	basicstyle={\ttfamily\scriptsize},
	commentstyle=\color{magenta},
	language=Python,
	morekeywords={True,False},
	tabsize=4,
	stringstyle=\color{green!55!black},
	frame=single,
	keywordstyle=\color{blue},
	showstringspaces=false,
	columns=fullflexible,
	keepspaces=true}

\newcommand\numberthis{\addtocounter{equation}{1}\tag{\theequation}}
\newcommand{\ihat}{\boldsymbol{\hat{\textbf{\i}}}}
\newcommand{\jhat}{\boldsymbol{\hat{\textbf{\j}}}}
\newcommand{\khat}{\boldsymbol{\hat{\textbf{k}}}}
\newcommand{\del}[1]{\textbf{#1)}}
\newcommand{\svar}[1]{\underline{\underline{{#1}}}}


\begin{document}



\title{FYS3150 - Project 1}
\date{\today}
\author{Sigurd Sørlie Rustad and Vegard Falmår}


\newpage

\begin{abstract}
Abstract om du vil
\end{abstract}
\maketitle                                % creates the title, author, date & abstract

\section{introduction}

We can describe the evolution of many physical systems with the help of differential equations, but because of their complexity we  are unable to solve these equations to find an analytical solution. Therefore we have to use numerical methods on computers in order to approximate the solution. Computers are limited in both their memory and accuracy so we have to be careful when both selecting the numerical method and how we implement it. In this report we are going to explore these issues by trying to solve the one-dimensional Poisson equation with Dirichlet boundary conditions, described in equation \ref{eq:Poisson_1D} in the theory section.

The example function we are going to use in our studies is described by the equation (\ref{eq:f}) and has an analytical solution (\ref{eq:u}) we can use to compare our results.
\begin{equation}
	-\frac{d^2u}{dx^2} = f(x) = 100e^{-10x}
	\label{eq:f}
\end{equation}
\begin{equation}
	u(x) = 1 - (1 - e^{-10})x - e^{-10x}
	\label{eq:u}
\end{equation}
In order to solve equation (\ref{eq:f}) we end up with a set of linear equations described by a tridiagonal matrix multiplied with a vector (see the theory section for further explanation). Now there are many ways we can solve these equations, each with their own pros and cons. The methods we are going to explore is one where we solve for a general tridiagonal matrix, one where we specialize the algorithm to our tridiagonal matrix and lastly by using LU decomposition.

\section{Theory}

\subsection{Matrix formulation of the discrete one-dimensional Poisson equation}
The one-dimensional Poisson equation with Dirichlet boundary conditions is given by equation \ref{eq:Poisson_1D}.
\begin{equation}
  \label{eq:Poisson_1D}
  - \frac{\mathrm d^2 u(x)}{\mathrm d x^2} = f(x), \quad x \in (0, 1), \quad u(0) = u(1) = 0
\end{equation}
We defnine the discretized approximation to $u$ to be $v_i$ at points $x_i = i h$ evenly spaced between $x_0 = 0$ and $x_{n+1} = 1$. The step length between the points is $h = 1/(n + 1)$. The boundary conditions from equation \ref{eq:Poisson_1D} then give $v_0 = v_{n+1} = 0$. An approximation to the second derivative of $u$, derived from the Taylor expansion, is
\begin{equation}
  \label{eq:discrete_2nd_deriavative}
  \frac{-v_{i-1} + 2 v_i - v_{i+1}}{h^2} = f_i \quad for i = 1, 2, ..., n
\end{equation}
where $f_i = f(x_i)$.

Written out for all $i$, equation \ref{eq:discrete_2nd_deriavative} becomes

\begin{align*}
  - v_0 + 2 v_1 - v_2 &= h^2 f_1 \\
  - v_1 + 2 v_2 - v_3 &= h^2 f_2 \\
  ... \\
  - v_{n-2} + 2 v_{n-1} - v_n &= h^2 f_{n-1} \\
  - v_{n-1} + 2 v_n - v_{n+1} &= h^2 f_n \\
\end{align*}
In general, this can be rearranged slightly so that
\begin{align*}
  2 v_1 - v_2 &= h^2 f_1 + v_0 \\
  - v_1 + 2 v_2 - v_3 &= h^2 f_2 \\
  ... \\
  - v_{n-2} + 2 v_{n-1} - v_n &= h^2 f_{n-1} \\
  - v_{n-1} + 2 v_n &= h^2 f_n + v_{n+1} \\
\end{align*}
This system of equations can be written in matrix form as
\begin{equation}
  \label{eq:Poisson_1D_matrix}
  \boldsymbol A \boldsymbol v = \boldsymbol{\tilde{b}},
\end{equation}
explicitly
\begin{equation*}{}
  \begin{bmatrix}
2  & -1 & 0  & 0 & ... & 0 \\
-1 & 2  & -1 & 0 & ... & 0 \\
.  &    &    &   &     & \\
.  &    &    &   &     & \\
.  &    &    &   &     & \\
0  & ...& 0  &-1 & 2   & -1  \\
0  & ...& 0  & 0 &-1 & 2  \\
\end{bmatrix}
  \begin{bmatrix}
 v_1  \\
 v_2  \\
.   \\
.   \\
.   \\
 v_{n-1} \\
v_n \\
\end{bmatrix}
=
\begin{bmatrix}
h^2 f_1 + v_0 \\
h^2 f_2  \\
.   \\
.   \\
.   \\
h^2 f_{n-1} \\
h^2 f_n + v_{n+1}\\
\end{bmatrix}
\end{equation*}
With $v_0 = v_{n+1} = 0$, the right side reduces to $\tilde b_i = h^2 f_i$.

\subsection{Solve tridiagonal matrix equation}

In order to solve the tridiagonal matrix below we need to develop an algorithm. As mentioned in the exercise set \citep{oppgavetekst} we first need to do a decomposition and forward substitution.
\begin{equation*}
\mathbf{A}\mathbf{v} = \begin{bmatrix}
b_1& c_1 & 0 &\dots   & \dots &\dots \\
a_1 & b_2 & c_2 &\dots &\dots &\dots \\
& a_2 & b_3 & c_3 & \dots & \dots \\
& \dots   & \dots &\dots   &\dots & \dots \\
&   &  &a_{n-2}  &b_{n-1}& c_{n-1} \\
&    &  &   &a_{n-1} & b_n \\
\end{bmatrix}\begin{bmatrix}
v_1\\
v_2\\
\dots \\
\dots  \\
\dots \\
v_n\\
\end{bmatrix}
=\begin{bmatrix}
\tilde{b}_1\\
\tilde{b}_2\\
\dots \\
\dots \\
\dots \\
\tilde{b}_n\\
\end{bmatrix}.
\end{equation*}

Looking at the first matrix multiplication we get the following expression.

\begin{equation}
	b_1 v_1 + c_1 v_2 = \tilde{b} \implies v_1 + \alpha_1 v_2 = \rho _1, \ \ \alpha_1 = \frac{c_1}{b_1} \wedge \rho_1 = \frac{\tilde{b}_1}{b_1} 
	\label{eq:mat1}
\end{equation}

Doing the second matrix multiplication we get

\begin{equation}
	a_1 v_1 + b_2 v_2 + c_2 v_3 = \tilde{b}_2
	\label{eq:mat2}
\end{equation}

If we multiply equation \ref{eq:mat1} by $a_1$, and subtract it from equation \ref{eq:mat2} the resulting expression becomes

\begin{align*}
	(b_2 - \alpha_1 a_1) v_2 + c_2 v_3 &= \tilde{b}_2 - \rho_1 a_1 \\
	\implies v_2 + \frac{c_2}{b_2 - \alpha_1 a_1}v_3 &= \frac{\tilde{b}_2 - \rho_1 a_1}{b_2 - \alpha_1 a_1} \\
	\implies v_2 + \alpha_2 v_3 &= \rho_2 \\
	\text{where}\ \ \alpha_2 = \frac{c_2}{b_2 - \alpha_1 a_1} &\wedge \rho_2 = \frac{\tilde{b}_2 - \rho_1 a_1}{b_2 - \alpha_1 a_1}
\end{align*}

Noticing the pattern in $\rho$ and $\alpha$ we can generalize the terms.

\begin{equation}
	\alpha_i = \frac{c_i}{b_i - \alpha_{i-1}a_{i-1}} \ \ \text{for} \ \ i = 2, 3, ..., n-1
	\label{eq:alpha}
\end{equation}
\begin{equation}
	\rho_i = \frac{\tilde{b}_{i} - \rho_{i-1} a_{i-1}}{b_{i} - \alpha_{i-1} a_{i-1}}\ \ \text{for} \ \ i = 2, 3, ..., n
	\label{eq:rho}
\end{equation}

Inserting the terms into the matrix above, we get a much simpler set of equations.

\begin{equation*}
	\mathbf{A}\mathbf{v} = \begin{bmatrix}
		1& \alpha_1 & 0 &\dots   & \dots &\dots \\
		0 & 1 & \alpha_2 &\dots &\dots &\dots \\
		& 0 & 1 & \alpha_3 & \dots & \dots \\
		& \dots   & \dots &\dots   &\dots & \dots \\
		&   &  &0  &1& \alpha_{n-1} \\
		&    &  &   &0 & 1 \\
	\end{bmatrix}\begin{bmatrix}
		v_1\\
		v_2\\
		\dots \\
		\dots  \\
		\dots \\
		v_n\\
	\end{bmatrix}
	=\begin{bmatrix}
		\rho_1\\
		\rho_2\\
		\dots \\
		\dots \\
		\dots \\
		\rho_n\\
	\end{bmatrix}.
\end{equation*}

Now the last step is to do a backward substitution. Starting with $v_n = \rho_n$ we can work our way backward, with the general expression
\begin{equation}
	v_{i-1} = \rho_{i-1} - \alpha_{i-1}v_i \ \ \text{for} \ \ i = n, n-1, ..., 2
\end{equation}

Now in this report we are going to consider a matrix with elements $b_n = 2$ and $a_n = c_n = -1$. We can insert this into equations (\ref{eq:alpha}) and (\ref{eq:rho}) to get the expressions (\ref{eq:alpha_spes}) and (\ref{eq:rho_spes}).
\begin{equation}
\alpha_i = \frac{-1}{2 + \alpha_{i-1}}
\label{eq:alpha_spes}
\end{equation}
\begin{equation}
	\rho_i = \frac{\tilde{b} + \rho_{i-1}}{2 + \alpha_{i-1}}
	\label{eq:rho_spes}
\end{equation}

\section{Method}

First numerical test 

\section{Results}

\section{Discussion}

\section{Conclution}

\onecolumngrid
\vspace{1cm} % some extra space
\newpage
\begin{thebibliography}{}
\bibitem[Engeland, Hjorth-Jensen, Viefers, Raklev og Flekkøy (2020)]{Kompendium81} Engeland, Hjorth-Jensen, Viefers, Raklev og Flekkøy,  2020, \textit{Kompendium i FYS2140 Kvantefysikk, Versjon 3}, s. 81
\bibitem[Griffiths (2018)]{Griffiths44} Griffiths, D. J, Schroeter, D. F.,  2018, \textit{Introduction to Quantum Mechanics, Third edition}, s. 44
\bibitem {oppgavetekst} HER REFERERERERERER VI TIL OPPGAVETEKSTEN
\end{thebibliography}

\section{Appendix}


\section{Kode}

% \subsection{\textit{main.py}}
% \lstinputlisting[language=Python]{../main.py}
%
% \subsection{Output fra \textit{main.py}}
% \verbatiminput{../main_output.txt}
%
% \subsection{\textit{oppg1.py}} \label{kode:oppg1.py}
% \lstinputlisting[language=Python]{../code/oppg1.py}
%
% \subsection{\textit{oppg2.py}} \label{kode:oppg2.py}
% \lstinputlisting[language=Python]{../code/oppg2.py}
%
% \subsection{\textit{oppg3.py}} \label{kode:oppg3.py}
% \lstinputlisting[language=Python]{../code/oppg3.py}
%
% \subsection{\textit{oppg4.py}} \label{kode:oppg4.py}
% \lstinputlisting[language=Python]{../code/oppg4.py}
%
% \subsection{\textit{appendix.py}} \label{kode:appendix.py}
% \lstinputlisting[language=Python]{../code/appendix.py}



\end{document}
