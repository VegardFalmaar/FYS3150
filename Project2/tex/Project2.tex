\documentclass[reprint,english,notitlepage]{revtex4-1}  % defines the basic parameters of the document
% if you want a single-column, remove reprint

% allows special characters (including æøå)
\usepackage[utf8]{inputenc}
\usepackage [norsk]{babel} %if you write norwegian

\usepackage{makecell}       %denne gjør at man kan ha nye linjer

\usepackage{physics,amssymb}  % mathematical symbols (physics imports amsmath)
\usepackage{graphicx}         % include graphics such as plots
\usepackage{xcolor}           % set colors
\usepackage{hyperref}         % automagic cross-referencing (this is GODLIKE)
\usepackage{tikz}             % draw figures manually
\usepackage{listings}         % display code
\usepackage{subfigure}        % imports a lot of cool and useful figure commands

% defines the color of hyperref objects
% Blending two colors:  blue!80!black  =  80% blue and 20% black
\hypersetup{ % this is just my personal choice, feel free to change things
    colorlinks,
    linkcolor={red!50!black},
    citecolor={blue!50!black},
    urlcolor={blue!80!black}}

%% Defines the style of the programming listing
%% This is actually my personal template, go ahead and change stuff if you want
\lstset{ %
	inputpath=,
	backgroundcolor=\color{white!88!black},
	basicstyle={\ttfamily\scriptsize},
	commentstyle=\color{magenta},
	language=Python,
	morekeywords={True,False},
	tabsize=4,
	stringstyle=\color{green!55!black},
	frame=single,
	keywordstyle=\color{blue},
	showstringspaces=false,
	columns=fullflexible,
	keepspaces=true}


\begin{document}


\title{FYS3150 - Project 2}
\date{\today}               
\author{Vegard Falmår and Sigurd Sørlie Rustad}
\affiliation{Universitetet i Oslo}

\email{sigurdsr@gmail.com}
\email{vegard@mail.no}
\newpage

\begin{abstract}

\end{abstract}
\maketitle


\section{Introduction}

The harmonic oscillator (HO) potential appears in many areas within quantum physics. This is because many potentials can be approximated, at least in some intervals, with a HO potential. The most famous example of this might be the model for a hydrogen atom, where the electron is bound to the proton by a HO potential. That is why, in this report, we are going to study both one and multiple electrons in a HO potential. We are also going to account for electrical forces between the particles.

 

\section{Theory}

\subsection{Unitary transformation}
The transformed of a unitary matrix ($U$) is its inverse.
\begin{equation*}
	U^T = U^{-1}
\end{equation*}
From this we can prove that a unitary transformation preserves the orthonormality of vectors. Consider the set of orthonormal vectors $\{ \mathbf{v}_i \}_i$ and the unitary transformation $\{ U\mathbf{v}_i \}_i = \{ \mathbf{w}_i \}_i$.
\begin{align*}
	\mathbf{w}_i^T\mathbf{w}_j &= (U\mathbf{v}_i)^TU\mathbf{v}_j \\
	&= \mathbf{v}_i^TU^TU\mathbf{v}_j = \mathbf{v}_i^T\mathbf{v}_j \\
	&= \delta_{i,j}
\end{align*}
We notice that orthonormality is perserved.

\subsection{Jacobi's rotation algorithm}
Jacobi's rotation algorithm uses unitary transformations to diagonalize a matrix. A detailed description of the algorithm can be found here \citep{lecnotes}, however we will describe it briefly. In order to diagonalize a given matrix $A$, as mentioned over, we perform a series of unitary transformations.
\begin{equation*}
	B = U_n^T U_{n-1}^T ... U_0^T A U_0 ... U_{n-1}U_n
\end{equation*}
Here $U_i$ are the unitary matrices and $B$ the resulting diagonal matrix. An example of how $U_i$ can look is given under.
\begin{equation*}
	  \begin{bmatrix}
	1  & 0 & ...  & 0 & 0 & ... & 0 & 0 \\
	0 & 1  & ... & 0 & 0 & ... & 0 & 0 \\
	...  & ... & ... & ... & ... & ... & 0 & ... \\
	0  & 0 & ... & \cos \theta & 0 & ... & 0 & \sin \theta \\
	0  & 0 & ... & 0 & 1 & ... & 0 & 0 \\
	...  & ...& ...  &... & ...   & ... &1 &...  \\
	0  & 0& ...  & -\sin \theta &0 & ... &0 &\cos \theta  \\
	\end{bmatrix}
\end{equation*}
The geometric interpretation is that $U_i$ performs a rotation on $T$ in order to zero out non-diagonal elements. It turns out that the fastest way to do this, is to zero out the largest non-diagonal matrix-element. First We define:
\begin{equation*}
	\cot(\theta) = \tau = \frac{a_{ll} - a_{kk}}{2a_{kl}},
\end{equation*}
Where $a_{kl}$ is the largest non-diagonal element in $A$. Now to shorten notation we use $\tan(\theta) = t = s/c$, where $s = \sin(\theta) \wedge c = \cos(\theta)$. By defining $\theta$ such that $a_{kl}$ becomes zero we get the quadratic equation
\begin{equation*}
	t^2 + 2\tau t -1 = 0 \implies t = -\tau \pm \sqrt{1+\tau^2},
\end{equation*}
and can also obtain $c$ and $s$
\begin{equation*}
	c = \frac{1}{1+t^2} \wedge s = tc.
\end{equation*}
The actual transformation is defined by the equations
\begin{align*}
	b_{ik} &= a_{ik}c - a_{il}s, i \neq k, i \neq l \\ 
	b_{il} &= a_{il}c + a_{ik}s, i \neq k, i \neq l \\
	b_{kk} &= a_{kk}c^2 - 2a_{kl}cs + a_{ll}s^2 \\
	b_{ll} &= a_{ll}c^2 + 2a_{kl}cs + a_{kk}s^2 \\
	b_{kl} &= (a_{kk} - a_{ll} )cs + a_{kl}(c^2 - s^2)
\end{align*}
Where $b_{ij}$ are elements in $B$, again see \citep{lecnotes} for a more detailed description of the algorithm.

\subsection{Electrons in a harmonic oscillator potential}

It turns out that if you try to solve Schrodinger's equation you end up with a eigenvalue problem. We will derive the math here, but everything is taken from here \citep{oppgavetekst}. First we consider the Schrodinger's equation for one electron, we then, because of spherical symmetry, only need to look at the radial part.
\begin{equation}
	-\frac{\hbar^2}{2m}\bigg(\frac{1}{r^2}\frac{d}{dr}r^2\frac{d}{dr}-\frac{l(l+1)}{r^2}\bigg)R(r) + V(r) R(r) = ER(r)
	\label{eq:SE}
\end{equation}
Where $R(r)$ is the radial wave equation, $V(r)$ the potential, $l$ orbital momentum and $r$ distance from the origin. Now because we only have one electron we end up with the HO potential $V(r) = (1/2)kr^2$. Substituting $R(r) = (1/r)u(r)$ and $\rho = (1/\alpha)r$ (where $\alpha$ is a constant with dimension length) we obtain
\begin{equation*}
	-\frac{\hbar^2}{2m\alpha^2}\frac{d^2}{d\rho^2}u(\rho) + \left(V(\rho) + \frac{l(l+1)}{\rho^2}\frac{\hbar^2}{2m\alpha^2}\right)u(\rho) = Eu(\rho).
\end{equation*}
Setting the orbital momentum $l=0$, inserting $V(\rho) = (1/2)k\alpha^2\rho^2$ and defining
\begin{equation*}
	\alpha \equiv \left(\frac{\hbar^2}{mk}\right)^{1/4} \wedge \lambda \equiv \frac{2m\alpha^2}{\hbar}E
\end{equation*}
Schrodinger's equation can then be described by the simple equation (\ref{eq:SE1}).
\begin{equation}
	-\frac{d^2}{d\rho^2}u(\rho)+\rho^2u(\rho) = \lambda u(\rho)
	\label{eq:SE1}
\end{equation}
Now we can expand our model to two particles with Coulomb potential between them. First introducing the relative position $\mathbf{r}$ and center-of-mass coordinate $\mathbf{R}$
\begin{equation*}
	\mathbf{r} = \mathbf{r_1} - \mathbf{r_2} \wedge \mathbf{R} = \frac{1}{2(\mathbf{r_1}+\mathbf{r_2})}.
\end{equation*}
Where $\mathbf{r_1}$ and $\mathbf{r_2}$ are the particles positions.Then the radial Schrodinger's becomes
\begin{equation*}
	\left(  -\frac{\hbar^2}{m} \frac{d^2}{dr^2} -\frac{\hbar^2}{4 m} \frac{d^2}{dR^2}+ \frac{1}{4} k r^2+  kR^2\right)u(r,R)  = E^{(2)} u(r,R).
\end{equation*}
$E^{(2)}$ denotes the fact that we are looking at the total energy of two particles. From an ansatz for the wave equation, we can perform separation of variable on $r$ and $R$.
\begin{equation*}
	u(r, R) = \psi(r)\phi(R) \wedge E^{(2)} = E_r + E_R
\end{equation*}
Adding the Coulomb potential
\begin{equation*}
	\frac{\beta e^2}{r}
\end{equation*}
between the two electrons we get
\begin{equation*}
	\left(  -\frac{\hbar^2}{m} \frac{d^2}{dr^2}+ \frac{1}{4}k r^2+\frac{\beta e^2}{r}\right)\psi(r)  = E_r \psi(r).
\end{equation*}
Again introducing $\rho = r/\alpha$ and defining a few variables
\begin{equation*}
	\omega_r^2\equiv \frac{1}{4}\frac{mk}{\hbar^2} \alpha^4 \wedge \alpha \equiv \frac{\hbar^2}{m\beta e^2} \wedge \lambda \equiv \frac{m\alpha^2}{\hbar^2}E
\end{equation*}
We arrive at the Schrodinger's equation (\ref{eq:SE2})
\begin{equation}
	-\frac{d^2}{d\rho^2}\psi(\rho) + \omega_r^2\rho^2\psi(\rho) + \frac{1}{\rho} = \lambda\psi(\rho)
	\label{eq:SE2}
\end{equation}

\section{Method}

\section{Results}

\section{Discussion}

\section{Conclusion}

\onecolumngrid
\begin{thebibliography}{}
\bibitem{lecnotes} http://compphysics.github.io/
ComputationalPhysics/doc/pub/eigvalues/html/.\_eigvalues-bs011.html
\bibitem{oppgavetekst} OPPGAVETEKST REF
\end{thebibliography}
\end{document}