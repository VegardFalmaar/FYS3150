\documentclass[reprint,english,notitlepage]{revtex4-1}  % defines the basic parameters of the document
% if you want a single-column, remove reprint

% allows special characters (including æøå)
\usepackage[utf8]{inputenc}
\usepackage [norsk]{babel} %if you write norwegian

\usepackage{makecell}       %denne gjør at man kan ha nye linjer

\usepackage{physics,amssymb}  % mathematical symbols (physics imports amsmath)
\usepackage{graphicx}         % include graphics such as plots
\usepackage{xcolor}           % set colors
\usepackage{hyperref}         % automagic cross-referencing (this is GODLIKE)
\usepackage{tikz}             % draw figures manually
\usepackage{listings}         % display code
\usepackage{subfigure}        % imports a lot of cool and useful figure commands

% defines the color of hyperref objects
% Blending two colors:  blue!80!black  =  80% blue and 20% black
\hypersetup{ % this is just my personal choice, feel free to change things
    colorlinks,
    linkcolor={red!50!black},
    citecolor={blue!50!black},
    urlcolor={blue!80!black}}

%% Defines the style of the programming listing
%% This is actually my personal template, go ahead and change stuff if you want
\lstset{ %
	inputpath=,
	backgroundcolor=\color{white!88!black},
	basicstyle={\ttfamily\scriptsize},
	commentstyle=\color{magenta},
	language=Python,
	morekeywords={True,False},
	tabsize=4,
	stringstyle=\color{green!55!black},
	frame=single,
	keywordstyle=\color{blue},
	showstringspaces=false,
	columns=fullflexible,
	keepspaces=true}


\begin{document}


\title{FYS3150 - Project 2}
\date{\today}               
\author{Vegard Falmår and Sigurd Sørlie Rustad}
\affiliation{Universitetet i Oslo}

\email{sigurdsr@gmail.com}
\email{vegard@mail.no}
\newpage

\begin{abstract}

\end{abstract}
\maketitle


\section{Introduction}

\section{Theory}

\subsection{Unitary transformation}
The transformed of a unitary matrix ($U$) is its inverse.
\begin{equation*}
	U^T = U^{-1}
\end{equation*}
From this we can prove that a unitary transformation preserves the orthonormality of vectors. Consider the set of orthonormal vectors $\{ \mathbf{v}_i \}_i$ and the unitary transformation $\{ U\mathbf{v}_i \}_i = \{ \mathbf{w}_i \}_i$.
\begin{align*}
	\mathbf{w}_i^T\mathbf{w}_j &= (U\mathbf{v}_i)^TU\mathbf{v}_j \\
	&= \mathbf{v}_i^TU^TU\mathbf{v}_j = \mathbf{v}_i^T\mathbf{v}_j \\
	&= \delta_{i,j}
\end{align*}
We notice that orthonormality is perserved.

\subsection{Jacobi's rotation algorithm}
Jacobi's rotation algorithm uses unitary transforms to diagonalize a matrix and preserves eigenvalues. A detailed description of the algorithm can be found here \citep{lecnotes}, however we will describe it briefly here. In order to diagonalize a given matrix $A$, as mentioned over, we perform a series of unitary transformations.
\begin{equation*}
	B = U_n^T U_{n-1}^T ... U_0^T A U_0 ... U_{n-1}U_n
\end{equation*}
Here $U_i$ are the unitary matrices and $B$ the resulting diagonal matrix. The geometric interpretation is that $U_i$ performs a rotation on $T$ in order to zero out elements. It turns out that the fastest way to do this, is to zero out the largest non-diagonal matrix-element. First We define:
\begin{equation*}
	\cot(\theta) = \tau = \frac{a_{ll} - a_{kk}}{2a_{kl}}.
\end{equation*}
Now to shorten notation we use $\tan(\theta) = t = s/c$, where $s = \sin(\theta) \wedge c = \cos(\theta)$. By defining $\theta$ such that $a_{kl}$ becomes zero we get the quadratic equation
\begin{equation*}
	t^2 + 2\tau t -1 = 0 \implies t = -\tau \pm \sqrt{1+\tau^2},
\end{equation*}
and can also obtain $c$ and $s$
\begin{equation*}
	c = \frac{1}{1+t^2} \wedge s = tc.
\end{equation*}
The actual transformation is defined by the equations
\begin{align*}
	b_{ik} &= a_{ik}c - a_{il}s, i \neq k, i \neq l \\ 
	b_{il} &= a_{il}c + a_{ik}s, i \neq k, i \neq l \\
	b_{kk} &= a_{kk}c^2 - 2a_{kl}cs + a_{ll}s^2 \\
	b_{ll} &= a_{ll}c^2 + 2a_{kl}cs + a_{kk}s^2 \\
	b_{kl} &= (a_{kk} - a_{ll} )cs + a_{kl}(c^2 - s^2)
\end{align*}
Again see \citep{lecnotes} for a more detail description of the algorithm.

\section{Method}

\section{Results}

\section{Discussion}

\section{Conclusion}

\onecolumngrid
\begin{thebibliography}{}
\bibitem{lecnotes} http://compphysics.github.io/
ComputationalPhysics/doc/pub/eigvalues/html/.\_eigvalues-bs011.html
\end{thebibliography}
\end{document}