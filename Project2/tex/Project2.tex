\documentclass[reprint,english,notitlepage]{revtex4-1}  % defines the basic parameters of the document
% if you want a single-column, remove reprint

% allows special characters (including æøå)
\usepackage[utf8]{inputenc}
\usepackage [norsk]{babel} %if you write norwegian

\usepackage{makecell}       %denne gjør at man kan ha nye linjer

\usepackage{physics,amssymb}  % mathematical symbols (physics imports amsmath)
\usepackage{graphicx}         % include graphics such as plots
\usepackage{xcolor}           % set colors
\usepackage{hyperref}         % automagic cross-referencing (this is GODLIKE)
\usepackage{tikz}             % draw figures manually
\usepackage{listings}         % display code
\usepackage{subfigure}        % imports a lot of cool and useful figure commands

% defines the color of hyperref objects
% Blending two colors:  blue!80!black  =  80% blue and 20% black
\hypersetup{ % this is just my personal choice, feel free to change things
    colorlinks,
    linkcolor={red!50!black},
    citecolor={blue!50!black},
    urlcolor={blue!80!black}}

%% Defines the style of the programming listing
%% This is actually my personal template, go ahead and change stuff if you want
\lstset{ %
	inputpath=,
	backgroundcolor=\color{white!88!black},
	basicstyle={\ttfamily\scriptsize},
	commentstyle=\color{magenta},
	language=Python,
	morekeywords={True,False},
	tabsize=4,
	stringstyle=\color{green!55!black},
	frame=single,
	keywordstyle=\color{blue},
	showstringspaces=false,
	columns=fullflexible,
	keepspaces=true}


\begin{document}


\title{FYS3150 - Project 2}
\date{\today}               
\author{Vegard Falmår and Sigurd Sørlie Rustad}
\affiliation{Universitetet i Oslo}

\email{sigurdsr@gmail.com}
\email{vegard@mail.no}
\newpage

\begin{abstract}

\end{abstract}
\maketitle


\section{Introduction}

\section{Theory}

\subsection{Unitary transformation}

The transformed of a unitary matrix ($U$) is its inverse.
\begin{equation*}
	U^T = U^{-1}
\end{equation*}
From this we can prove that a unitary transformation preserves the orthonormality of vectors. Consider the set of orthonormal vectors $\{ \mathbf{v}_i \}_i$ and the unitary transformation $\{ U\mathbf{v}_i \}_i = \{ \mathbf{w}_i \}_i$.
\begin{align*}
	\mathbf{w}_i^T\mathbf{w}_j &= (U\mathbf{v}_i)^TU\mathbf{v}_j \\
	&= \mathbf{v}_i^TU^TU\mathbf{v}_j = \mathbf{v}_i^T\mathbf{v}_j \\
	&= \delta_{i,j}
\end{align*}

\section{Method}

\section{Results}

\section{Discussion}

\section{Conclusion}

\begin{acknowledgements}
\end{acknowledgements}

\begin{thebibliography}{}
\bibitem{kompendium} N. J. Edin, K. E. Pitman, 2020, KOMPENDIUM FYS2150
\end{thebibliography}
\end{document}