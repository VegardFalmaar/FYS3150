\documentclass[reprint, english,notitlepage,nofootinbib]{revtex4-1}  % defines the basic parameters of the document
% if you want a single-column, remove reprint

% allows special characters (including æøå)
\usepackage[utf8]{inputenc}
\usepackage [norsk]{babel} %if you write norwegian
%\usepackage[english]{babel}  %if you write english


%% note that you may need to download some of these packages manually, it depends on your setup.
%% I recommend downloading TeXMaker, because it includes a large library of the most common packages.

\usepackage{physics,amssymb}  % mathematical symbols (physics imports amsmath)
\usepackage{graphicx}         % include graphics such as plots
\usepackage{xcolor}           % set colors
\usepackage{hyperref}         % automagic cross-referencing (this is GODLIKE)
\usepackage{tikz}             % draw figures manually
\usepackage{listings}         % display code
\usepackage{subfigure}        % imports a lot of cool and useful figure commands
\usepackage{verbatim}
\usepackage{adjustbox}


% defines the color of hyperref objects
% Blending two colors:  blue!80!black  =  80% blue and 20% black
\hypersetup{ % this is just my personal choice, feel free to change things
    colorlinks,
    linkcolor={red!50!black},
    citecolor={blue!50!black},
    urlcolor={blue!80!black}}

%% Defines the style of the programming listing
%% This is actually my personal template, go ahead and change stuff if you want
\lstset{ %
	inputpath=,
	backgroundcolor=\color{white!88!black},
	basicstyle={\ttfamily\scriptsize},
	commentstyle=\color{magenta},
	language=Python,
	morekeywords={True,False},
	tabsize=4,
	stringstyle=\color{green!55!black},
	frame=single,
	keywordstyle=\color{blue},
	showstringspaces=false,
	columns=fullflexible,
	keepspaces=true}

\newcommand\numberthis{\addtocounter{equation}{1}\tag{\theequation}}
\newcommand{\ihat}{\boldsymbol{\hat{\textbf{\i}}}}
\newcommand{\jhat}{\boldsymbol{\hat{\textbf{\j}}}}
\newcommand{\khat}{\boldsymbol{\hat{\textbf{k}}}}
\newcommand{\del}[1]{\textbf{#1)}}
\newcommand{\svar}[1]{\underline{\underline{{#1}}}}
\newcommand{\vc}[1]{\mathbf{#1}}


\begin{document}



\title{FYS3150 - Project 3}
\date{\today}
\author{Vegard Falmår and Sigurd Sørlie Rustad}
\affiliation{Universitetet i Oslo}

\email{vegardfa@uio.no}
\email{sigurdsr@gmail.com}

\newpage

\begin{abstract}
Abstract om du vil
\end{abstract}
\maketitle                                % creates the title, author, date & abstract


\section{Introduction}



\section{Theory}

\subsection{Newton's laws and the motion of planets}

The well-known Newtons second law reads
\begin{equation}
  \label{eq:Newton_2nd}
  \sum \vc F = m \frac{\mathrm d^2 \vc r}{\mathrm d t^2},
\end{equation}
where $\sum \vc F$ is the sum of forces acting on an object, $m$ is its mass and $\vc r$ its position.

Newton's law of gravity says that the total gravitational force acting on an object $i$ from all the other objects in a system is
\begin{equation}
  \label{eq:Newton_grav}
  F_{G, i} = - G m_i \sum_{j \neq i} m_j \frac{\vc r_i - \vc r_j}{ \lvert \vc r_i - \vc r_j \rvert ^3}
\end{equation}
where $\vc r_i$ are the positions and $m_i$ are the masses of the objects. $G$ is the gravitational constant.

For the motions of planets, on which the gravitational force is the only force acting, the two equations \ref{eq:Newton_2nd} and \ref{eq:Newton_grav} combined give us the differential equation that describes the motion of an object:
\begin{equation}
  \label{eq:DE}
  \frac{\mathrm d^2 \vc r_i}{\mathrm d t^2} = \frac{1}{m_i} F_{G, i} = - G \sum_{j \neq i} m_j \frac{\vc r_i - \vc r_j}{ \lvert \vc r_i - \vc r_j \rvert ^3}
\end{equation}


\subsection{Velocity Verlet}

Classical problem of Newtonian mechanics often involve solvinga set of two coupled first order differential equations, namely
\begin{align*}
  \frac{\mathrm d x}{\mathrm dt} = v \\
  \frac{\mathrm d v}{\mathrm dt} = a
\end{align*}
where $x$ is position, $v$ is velocity and $a$ is acceleration. Doing a Taylor expansion of $x$ around a point in time $t = t_0$ gives
\begin{align*}
  x(t) &= x(t_0) + (t - t_0) \frac{\mathrm d x}{\mathrm dt}(t_0) + \frac{1}{2} (t - t_0)^2 \frac{\mathrm d^2 x}{\mathrm dt^2}(t_0) + O(h^3) \\
  &= x(t_0) + (t - t_0) v(t_0) + \frac{1}{2} (t - t_0)^2 a + O(h^3)
\end{align*}
An expression can then be found for position at a time $t = t_0 \pm h$:
\begin{equation}
  \label{eq:Taylor_exp_x}
  x(t_0 \pm h) = x(t_0) \pm h v(t_0) + \frac{1}{2} h^2 a \pm O(h^3)
\end{equation}

Discretizing equation \ref{eq:Taylor_exp_x} and letting $x_i = x(t)$, $x_{i+1} = x(t + h)$, $v_i = v(t)$ and $v_{i+1} = v(t + h)$, we get
\begin{align*}
  x_{i+1} \approx x_i + h v_i + \frac{h^2}{2} a_i \\
  v_{i+1} \approx v_i + h a_i + \frac{h^2}{2} \dot a_i,
\end{align*}
where $\dot a_i$ is the derivative of the acceleration with respect to time. Similarly to the case of Forward Euler, doing a Taylor expansion of $a$ gives after some manipulation
\begin{equation}
  \dot a_i \approx \frac{a_{i+1} - a_i}{h}
\end{equation}
Insering this into the expressions we have for $x_{i+1}$ and $v_{i+1}$ gives us the equations that describe the Velocity Verlet method:
\begin{align*}
  x_{i+1} \approx x_i + h v_i + \frac{h^2}{2} a_i \\
  v_{i+1} \approx v_i + \frac{h}{2} (a_{i+1} + a_i)
\end{align*}
From equation \ref{eq:Taylor_exp_x} we see that the mathematical error in this approximation goes like $O(h^3)$.


\subsection{Forward Euler}

From equation \ref{eq:Taylor_exp_x} we see that by including only the two first terms in the Taylor expansion of $x(t)$ and $v(t)$ we get
\begin{align*}
  x_{i+1} \approx x_i + h v_i \\
  v_{i+1} \approx v_i + h a_i
\end{align*}
This is referred to as the Forward Euler method of solving differential equations. The mathematical error in this approximation goes like $O(h^2)$.


\subsection{Conservation of angular momentum}

Kepler's second law states that if you draw a line from the Sun to a planet orbiting it, then that line would sweep out the same area in equal periods of time. We will use this law to derive the conservation of angular momentum.

For short periods of time $\mathrm d t$ the area swept out by the line from the Sun to the planet is approximately a triangle with area
\begin{equation*}
  A = \frac{1}{2} r v_\theta \mathrm dt
\end{equation*}
where $r$ is the distance from the planet to the sun and $v_\theta$ is the tangential velocity of the planet. Kepler's second law states that this area is constant for all intervalls of time of the same length. For this to be true, we must require
\begin{equation*}
  r v_\theta = \text{constant}
\end{equation*}
The angular momentum $\vc L$ of the planet around the Sun is
\begin{align*}
  \vc L &= \vc r \times \vc p \\
  &= r \ihat_r \times m(v_\theta \ihat_\theta + v_r \ihat_r) \\
  &= m r v_\theta \khat, \quad \text{as } \ihat_r \times \ihat_\theta = \khat \text{ and } \ihat_r \times \ihat_r = 0 \\
  &= \text{constant}
\end{align*}


\subsection{Escape velocity}

The inverse square law of gravity (Newton's law of gravity) is a conservative force with a potential
\begin{equation}
  \label{eq:pot_G}
  U_G(r) = - G \frac{m_1 m_2}{r}
\end{equation}
The sum of this potential energy and the kinetic energy will be conserved for celestial objects. In order to escape the gravitational pull of the Sun, a planet in orbit must have a total energy $E = U_G + K \ge 0$ as the potential energy goes to zero infinitely far away:
\begin{align*}
  E = U_G + K &\ge 0 \\
  K &\ge - U_G \\
  \frac{1}{2} M v^2 &\ge G \frac{M M_\odot}{r} \\
  v &\ge \sqrt{G \frac{2 M_\odot}{r}}
\end{align*}
A planet which begins at a distance 1 AU from the Sun, will therefore need an initial velocity of
\begin{align*}
  v_0 &= \sqrt{G \frac{2 M_\odot}{1 \text{ AU}}} \\
  &= \sqrt{8 \pi^2 \frac{\text{AU}^2}{\text{yr}^2}} \\
  &= 2 \sqrt 2 \pi \frac{\text{AU}}{\text{yr}} \numberthis \label{eq:escape_velocity}
\end{align*}
in order to be able to escape the gravitational pull of the Sun.



\section{Methods}

\subsection{Discretization of the equations of motion}

Equation \ref{eq:DE} is the equation of motion that describes the system of planets and stars in motion. We want to solve this equation numerically for all the planets in our solar system. In order to do this we must discretize the equations. We can rewrite the equation as a set of first order differential equations using the velocity:
\begin{align*}
   \frac{\mathrm d \vc r}{\mathrm d t} &= \vc v \\
   \frac{\mathrm d \vc v}{\mathrm d t} &= - G \sum_{j \neq i} m_j \frac{\vc r_i - \vc r_j}{ \lvert \vc r_i - \vc r_j \rvert ^3}
\end{align*}

The Verlet is not a self-starting algorithm.
Mainly applicable when we are not interested in the velocity
Velocity Verlet is energy conserving, Forward Euler is not


\subsection{Units of measurement}

Measuring distance in meters and time in seconds leads to very large numbers when doing calculations on the solar system. In order to do our calculations with numbers of magnitudes closer to $10^0$, we will measure time in years and distance in astronomical units AU, which is the mean distance between the Sun and the Earth. For an object in circular motion with radius $R$ and period $T$, the acceleration is given by the sentripetal acceleration
\begin{equation*}
  a = \frac{v^2}{r} = \frac{(2 \pi R / T)^2}{R} = \frac{4 \pi^2 R}{T^2}
\end{equation*}
With a small error, we can model the orbit of the Earth around the Sun as circular. To further simplify the expressions, we will measure masses in solar masses $M_\odot$. For the Earth's orbit we have $R = 1$ AU and $T = 1$ yr. This means that the force acting on the Earth from the Sun is
\begin{align*}
  F_{G, \text{Earth}} &= G \frac{M_{\text{Earth}} M_\odot}{\text{AU}^2} \\
  &= M_{\text{Earth}} a_{\text{Earth}} \\
  &\approx M_{\text{Earth}} \frac{4 \pi^2 \text{AU}}{\text{yr}^2} \\
  G &\approx 4 \pi^2 \frac{\text{AU}^3}{M_\odot \text{yr}^2}
\end{align*}
In units of AU, $M_\odot$ and yr the gravitational constant has a value of approximately $4 \pi$.





\onecolumngrid
\vspace{1cm} % some extra space
\newpage

\begin{thebibliography}{}
\bibitem[Engeland, Hjorth-Jensen, Viefers, Raklev og Flekkøy (2020)]{Kompendium81} Engeland, Hjorth-Jensen, Viefers, Raklev og Flekkøy,  2020, \textit{Kompendium i FYS2140 Kvantefysikk, Versjon 3}, s. 81
\bibitem[Griffiths (2018)]{Griffiths44} Griffiths, D. J, Schroeter, D. F.,  2018, \textit{Introduction to Quantum Mechanics, Third edition}, s. 44
\bibitem[Griffiths (2018)]{Griffiths75} Griffiths, D. J, Schroeter, D. F.,  2018, \textit{Introduction to Quantum Mechanics, Third edition}, s. 75
\bibitem[Griffiths (2018)]{Griffiths133} Griffiths, D. J, Schroeter, D. F.,  2018, \textit{Introduction to Quantum Mechanics, Third edition}, s. 133

\end{thebibliography}


\end{document}
