\documentclass[reprint, english,notitlepage,nofootinbib]{revtex4-1}  % defines the basic parameters of the document
% if you want a single-column, remove reprint

% allows special characters (including æøå)
\usepackage[utf8]{inputenc}
\usepackage [norsk]{babel} %if you write norwegian
%\usepackage[english]{babel}  %if you write english


%% note that you may need to download some of these packages manually, it depends on your setup.
%% I recommend downloading TeXMaker, because it includes a large library of the most common packages.

\usepackage{physics,amssymb}  % mathematical symbols (physics imports amsmath)
\usepackage{graphicx}         % include graphics such as plots
\usepackage{xcolor}           % set colors
\usepackage{hyperref}         % automagic cross-referencing (this is GODLIKE)
\usepackage{tikz}             % draw figures manually
\usepackage{listings}         % display code
\usepackage{subfigure}        % imports a lot of cool and useful figure commands
\usepackage{verbatim}
\usepackage{adjustbox}


% defines the color of hyperref objects
% Blending two colors:  blue!80!black  =  80% blue and 20% black
\hypersetup{ % this is just my personal choice, feel free to change things
    colorlinks,
    linkcolor={red!50!black},
    citecolor={blue!50!black},
    urlcolor={blue!80!black}}

%% Defines the style of the programming listing
%% This is actually my personal template, go ahead and change stuff if you want
\lstset{ %
	inputpath=,
	backgroundcolor=\color{white!88!black},
	basicstyle={\ttfamily\scriptsize},
	commentstyle=\color{magenta},
	language=Python,
	morekeywords={True,False},
	tabsize=4,
	stringstyle=\color{green!55!black},
	frame=single,
	keywordstyle=\color{blue},
	showstringspaces=false,
	columns=fullflexible,
	keepspaces=true}

\newcommand\numberthis{\addtocounter{equation}{1}\tag{\theequation}}
\newcommand{\ihat}{\boldsymbol{\hat{\textbf{\i}}}}
\newcommand{\jhat}{\boldsymbol{\hat{\textbf{\j}}}}
\newcommand{\khat}{\boldsymbol{\hat{\textbf{k}}}}
\newcommand{\del}[1]{\textbf{#1)}}
\newcommand{\svar}[1]{\underline{\underline{{#1}}}}
\newcommand{\vc}[1]{\mathbf{#1}}


\begin{document}



\title{FYS3150 - Project 3}
\date{\today}
\author{Vegard Falmår and Sigurd Sørlie Rustad}
\affiliation{Universitetet i Oslo}

\email{vegardfa@uio.no}
\email{sigurdsr@gmail.com}

\newpage

\begin{abstract}
Abstract om du vil
\end{abstract}
\maketitle                                % creates the title, author, date & abstract


\section{Introduction}

In this report we will study the solar system in three dimensions.


\section{Theory}

\subsection{Newton's laws and the motion of planets}

The well-known Newtons second law reads
\begin{equation}
  \label{eq:Newton_2nd}
  \sum \vc F = m \frac{\mathrm d^2 \vc r}{\mathrm d t^2},
\end{equation}
where $\sum \vc F$ is the sum of forces acting on an object, $m$ is its mass and $\vc r$ its position.

Newton's law of gravity says that the total gravitational force acting on an object $i$ from all the other objects in a system is
\begin{equation}
  \label{eq:Newton_grav}
  F_{G, i} = - G m_i \sum_{j \neq i} m_j \frac{\vc r_i - \vc r_j}{ \lvert \vc r_i - \vc r_j \rvert ^3}
\end{equation}
where $\vc r_i$ are the positions and $m_i$ are the masses of the objects. $G$ is the gravitational constant.

For the motions of planets, on which the gravitational force is the only force acting, the two equations \ref{eq:Newton_2nd} and \ref{eq:Newton_grav} combined give us the differential equation that describes the motion of an object:
\begin{equation}
  \label{eq:DE}
  \frac{\mathrm d^2 \vc r_i}{\mathrm d t^2} = \frac{1}{m_i} F_{G, i} = - G \sum_{j \neq i} m_j \frac{\vc r_i - \vc r_j}{ \lvert \vc r_i - \vc r_j \rvert ^3}
\end{equation}


\subsection{Velocity Verlet}

Classical problem of Newtonian mechanics often involve solvinga set of two coupled first order differential equations, namely
\begin{align*}
  \frac{\mathrm d x}{\mathrm dt} = v \\
  \frac{\mathrm d v}{\mathrm dt} = a
\end{align*}
where $x$ is position, $v$ is velocity and $a$ is acceleration. Doing a Taylor expansion of $x$ around a point in time $t = t_0$ gives
\begin{align*}
  x(t) &= x(t_0) + (t - t_0) \frac{\mathrm d x}{\mathrm dt}(t_0) + \frac{1}{2} (t - t_0)^2 \frac{\mathrm d^2 x}{\mathrm dt^2}(t_0) + O(h^3) \\
  &= x(t_0) + (t - t_0) v(t_0) + \frac{1}{2} (t - t_0)^2 a + O(h^3)
\end{align*}
An expression can then be found for position at a time $t = t_0 \pm h$:
\begin{equation}
  \label{eq:Taylor_exp_x}
  x(t_0 \pm h) = x(t_0) \pm h v(t_0) + \frac{1}{2} h^2 a \pm O(h^3)
\end{equation}

Discretizing equation \ref{eq:Taylor_exp_x} and letting $x_i = x(t)$, $x_{i+1} = x(t + h)$, $v_i = v(t)$ and $v_{i+1} = v(t + h)$, we get
\begin{align*}
  x_{i+1} \approx x_i + h v_i + \frac{h^2}{2} a_i \\
  v_{i+1} \approx v_i + h a_i + \frac{h^2}{2} \dot a_i,
\end{align*}
where $\dot a_i$ is the derivative of the acceleration with respect to time. Similarly to the case of Forward Euler, doing a Taylor expansion of $a$ gives after some manipulation
\begin{equation}
  \dot a_i \approx \frac{a_{i+1} - a_i}{h}
\end{equation}
Insering this into the expressions we have for $x_{i+1}$ and $v_{i+1}$ gives us the equations that describe the Velocity Verlet method:
\begin{align*}
  x_{i+1} \approx x_i + h v_i + \frac{h^2}{2} a_i \\
  v_{i+1} \approx v_i + \frac{h}{2} (a_{i+1} + a_i)
\end{align*}
From equation \ref{eq:Taylor_exp_x} we see that the mathematical error in this approximation goes like $O(h^3)$.


\subsection{Forward Euler}

From equation \ref{eq:Taylor_exp_x} we see that by including only the two first terms in the Taylor expansion of $x(t)$ and $v(t)$ we get
\begin{align*}
  x_{i+1} \approx x_i + h v_i \\
  v_{i+1} \approx v_i + h a_i
\end{align*}
This is referred to as the Forward Euler method of solving differential equations. The mathematical error in this approximation goes like $O(h^2)$.


\subsection{Conservation of angular momentum and energy}

Kepler's second law states that if you draw a line from the Sun to a planet orbiting it, then that line would sweep out the same area in equal periods of time. We will use this law to derive the conservation of angular momentum.

For short periods of time $\mathrm d t$ the area swept out by the line from the Sun to the planet is approximately a triangle with area
\begin{equation*}
  A = \frac{1}{2} r v_\theta \mathrm dt
\end{equation*}
where $r$ is the distance from the planet to the sun and $v_\theta$ is the tangential velocity of the planet. Kepler's second law states that this area is constant for all intervalls of time of the same length. For this to be true, we must require
\begin{equation*}
  r v_\theta = \text{constant}
\end{equation*}
The angular momentum $\vc L$ of the planet around the Sun is
\begin{align*}
  \vc L &= \vc r \times \vc p \\
  &= r \ihat_r \times m(v_\theta \ihat_\theta + v_r \ihat_r) \\
  &= m r v_\theta \khat, \quad \text{as } \ihat_r \times \ihat_\theta = \khat \text{ and } \ihat_r \times \ihat_r = 0 \\
  &= \text{constant}
\end{align*}

In this project we are going to be working with conservative forces. Therefore both energy and spin for our system should be conserved. When we are working with non-relativistic sizes, the total energy is given by the sum of kinetic and gravitational potential energy. Kinetic energy is given by
\begin{equation}
\label{eq:kinetic_energy}
K = \frac{1}{2}mv^2,
\end{equation}
where $K$ is the kinetic energy, $m$ is the mass and $v$ is the velocity. The potential energy of an object in a gravitational field is given by
\begin{equation}
\label{eq:potential_energy}
U = G\sum_{i}\frac{Mm_i}{r_i}
\end{equation}
Where $U$ is the potential energy, $G$ the gravitational constant, $r_i$ the relative distance between the objects, $m_i$ mass of an object creating the gravitational field, and $M$ mass of the object we want to find the energy of. Notice that the zero-point of potential energy is set to infinity.

\subsection{Escape velocity}

The inverse square law of gravity (Newton's law of gravity) is a conservative force with a potential
\begin{equation}
  \label{eq:pot_G}
  U_G(r) = - G \frac{m_1 m_2}{r}
\end{equation}
The sum of this potential energy and the kinetic energy will be conserved for celestial objects. In order to escape the gravitational pull of the Sun, a planet in orbit must have a total energy $E = U_G + K \ge 0$ as the potential energy goes to zero infinitely far away:
\begin{align*}
  E = U_G + K &\ge 0 \\
  K &\ge - U_G \\
  \frac{1}{2} M v^2 &\ge G \frac{M M_\odot}{r} \\
  v &\ge \sqrt{G \frac{2 M_\odot}{r}}
\end{align*}
A planet which begins at a distance 1 AU from the Sun, will therefore need an initial velocity of
\begin{align*}
  v_0 &= \sqrt{G \frac{2 M_\odot}{1 \text{ AU}}} \\
  &= \sqrt{8 \pi^2 \frac{\text{AU}^2}{\text{yr}^2}} \\
  &= 2 \sqrt 2 \pi \frac{\text{AU}}{\text{yr}} \numberthis \label{eq:escape_velocity}
\end{align*}
in order to be able to escape the gravitational pull of the Sun.

\subsection{Adjusting speed and position of the center of mass}
blah

\section{Methods}

\subsection{Discretization of the equations of motion}

Equation \ref{eq:DE} is the equation of motion that describes the system of planets and stars in motion. We want to solve this equation numerically for all the planets in our solar system. In order to do this we must discretize the equations. We can rewrite the equation as a set of first order differential equations using the velocity:
\begin{align*}
   \frac{\mathrm d \vc r}{\mathrm d t} &= \vc v \\
   \frac{\mathrm d \vc v}{\mathrm d t} &= - G \sum_{j \neq i} m_j \frac{\vc r_i - \vc r_j}{ \lvert \vc r_i - \vc r_j \rvert ^3}
\end{align*}

The Verlet is not a self-starting algorithm.
Mainly applicable when we are not interested in the velocity
Velocity Verlet is energy conserving, Forward Euler is not


\subsection{Units of measurement}

Measuring distance in meters and time in seconds leads to very large numbers when doing calculations on the solar system. In order to do our calculations with numbers of magnitudes closer to $10^0$, we will measure time in years and distance in astronomical units AU, which is the mean distance between the Sun and the Earth. For an object in circular motion with radius $R$ and period $T$, the acceleration is given by the sentripetal acceleration
\begin{equation*}
  a = \frac{v^2}{r} = \frac{(2 \pi R / T)^2}{R} = \frac{4 \pi^2 R}{T^2}
\end{equation*}
With a small error, we can model the orbit of the Earth around the Sun as circular. To further simplify the expressions, we will measure masses in solar masses $M_\odot$. For the Earth's orbit we have $R = 1$ AU and $T = 1$ yr. This means that the force acting on the Earth from the Sun is
\begin{align*}
  F_{G, \text{Earth}} &= G \frac{M_{\text{Earth}} M_\odot}{\text{AU}^2} \\
  &= M_{\text{Earth}} a_{\text{Earth}} \\
  &\approx M_{\text{Earth}} \frac{4 \pi^2 \text{AU}}{\text{yr}^2} \\
  G &\approx 4 \pi^2 \frac{\text{AU}^3}{M_\odot \text{yr}^2}
\end{align*}
In units of AU, $M_\odot$ and yr the gravitational constant has a value of approximately $4 \pi$.

\subsection{The data from NASA}
si litt om dataen fra nasa

\subsection{Testing the stability of the algorithms}
In order to make sure our algorithms runs correctly, we want to test it. Our first test will be to simulate the simplest possible system, namely earth orbiting the sun. We only look at the gravitational force from the sun on earth. We place the sun in origin and with zero velocity. The earth orbits with a distance of $r=1$AU away from the sun, and makes a full orbit every year. This gives it an initial velocity of $v = 2\pi \text{ AU}/\text{yr}.$ 
\\
Testing for different time steps $\Delta t$, we can look at the stability of velocity Verlet and Euler's forward method. Now we have to decide on what constitutes a stable orbit. The earth should have a constant distance away from the sun (1AU), so testing the distance away from the sun could be a good starting point. We will try to test for different $\Delta t$ and see how the distance varies from our theoretical value. We do each test over 10 years, and calculate the difference every $0.01/\Delta t$ steps (100 times).
\\
Another test will be conservation of kinetic and potential energy. As mentioned above, the earth should have a constant distance away from the sun, meaning potential energy is conserved. Earths speed should also be the same, meaning kinetic energy stays the same. Therefore, we will test for this as well. 
\\
The last test of our algorithm will be conservation of angular momentum. In the theory section, from Kepler's second law, we showed that angular momentum is conserved. Therefore we will also test whether that is the case for our Earth-Sun system. Adding some complexity, as well as testing it for the ideal orbit over, we tested conservation of angular momentum with the initial conditions collected from NASA (see \citep{NASA}). Still with the sun fixed in the origin, and only looking at the force from sun on the earth.

\subsection{Unit tests}
We are going to test several widely different systems. In order to reassure everything runs correct, we are going to have unit tests during simulations. We will test for conserved quantities, namely energy and angular momentum (see the theory section). The more we test and smaller tolerances, gives us more reassurance that our results are correct. This however, will be at the cost of computing time. We decide to test every $0.01/\Delta t$ step, giving us a total of 100 tests every simulation. The tolerance for energy and angular momentum have to depend on the simulation. For longer simulations we expect to see larger errors than for short ones.

\section{results}

\section{discussion}

discuss eventual differences between the verlet algorithm and the euler algorithm. Consider also the numver of flops involved and perform a timing of the two algorithm for equal final times.

\section{conclusion}




\onecolumngrid
\vspace{1cm} % some extra space
\newpage

\begin{thebibliography}{}
\bibitem[]{NASA} ref til nasa dataen

\end{thebibliography}


\end{document}
